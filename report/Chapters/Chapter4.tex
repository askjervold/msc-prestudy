% Chapter 4

\chapter{User Study}

\label{Chapter4}

\lhead{Chapter 4. \emph{User Study}}

%----------------------------------------------------------------------------------------
%	SECTION 1
%----------------------------------------------------------------------------------------

\section{Introduction}

This chapter describes the process for the user study to be performed. It will begin with a description of the user study design, followed by a section on each of the aspects we wish to improve in the existing application: learning, fun and motivation.

%----------------------------------------------------------------------------------------
%	SECTION 2
%----------------------------------------------------------------------------------------

\section{Experiment Design}

The user study will have a between-groups design. The application will be extended with the changes proposed in the previous chapter, resulting in a new version. Users will then be divided into two separate groups: one group will use the old (current) application, while the other will use the new, updated application. Both groups will go through the same procedure (to the extent that this is possible), but using different versions of the application. The groups will consist of subjects of both genders and of varied genders. These subjects will be recruited through the relevant museums and at the university.

The between-groups design allows for a relatively fast user study. The updated application can be implemented to its full extent before testing, and only one round of testing is required. By performing the testing at approximately the same time, it also becomes easier to make sure that subjects in both groups receive the same test with the same coverage, as projects are prone to change over time, which could affect groups tested at vastly different points in time.

%----------------------------------------------------------------------------------------
%	SECTION 3
%----------------------------------------------------------------------------------------

\section{Improved Learning}

To understand whether the changes to the application affect the learning potential of the game, subjects will be presented with a pen-and-paper test before and after using the application. The results from both of these tests will then be compared. The assumption is that subjects will have better results after playing the game. If the updated application provides a better learning environment, the subjects in the new-group should show a bigger improvement in the post-game test from the pre-game test than the old-group.

%----------------------------------------------------------------------------------------
%	SECTION 4
%----------------------------------------------------------------------------------------

\section{Increased Fun and Motivation}

To understand the impact on fun and motivation, we not only need to observe the subjects during play, but ask them about their experience. Additionally, the application will passively gather data during play. This data will show where users touch, and less interesting parts of the application will be less frequently visited. Observations during play will note player reactions and utterances and at what points during play they occur. After they are finished playing, a questionnaire will gauge them on their thoughts on both fun and motivation. Points that should be covered in the questionnaire include the following:
\begin{itemize}
	\item Was the game fun to play?
	\item Which mode was the most fun?
	\item Was the game too easy/too hard?
	\item Would you play the game during a regular visit to the museum?
	\item Did you feel motivated to play?
	\item If you felt motivated, what made you feel motivated?
	\item If you didn't feel motivated, what would have motivated you?
\end{itemize}