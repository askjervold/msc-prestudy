% Chapter 1

\chapter{Introduction}

\label{Chapter1}

\lhead{Chapter 1. \emph{Introduction}}

%----------------------------------------------------------------------------------------
%	SECTION 1
%----------------------------------------------------------------------------------------

\section{Introduction}

The goal of this paper is to plan the improvement of the existing Flora Danica application, a modifiable edu-game built on the Kivy framework\citep{Kivy}, designed for use on large multi-touch displays in informal learning situations. The paper will look into previous research and work on related topics such as edu-games, gamification, game elements, multi-touch wall displays and tabletops, and will use concepts from these areas to improve the existing application.

The first chapter provides an introduction to the paper, the existing application, edu-games and game elements. The second chapter explores and discusses previous work on related topics. The third chapter proposes a solution in possible extensions to the application, while the fourth chapter describes the planned process for conducting the user studies. The final chapter summarizes the findings and the way forward.

%----------------------------------------------------------------------------------------
%	SECTION 2
%----------------------------------------------------------------------------------------

\section{Existing Application}

The existing Flora Danica application that is being extended is a Python application built partially on the Kivy framework\citep{Kivy}. We are going to focus on improving the part built on the Kivy framework, which is a simple quiz game that can be played by two players either as a team ("team mode") or against each other ("versus mode").

During the game, the players will be presented with a series of questions, each with two alternatives for the right answer. In versus mode, each player answers the question separately, touching their answer button with one finger for the first alternative and two fingers for the second alternative. In team mode, the player to the left touches their answer button for the first alternative, while the player on the right touches their button for the second alternative.

The application allows for easy switching of content, and can be and has been modified for different institutions. Examples of different implementations of the application are those of the Norwegian Deaf Museum and of the Norwegian States Archive in Trondheim.

%----------------------------------------------------------------------------------------
%	SECTION 3
%----------------------------------------------------------------------------------------

\section{Edu-games}

Edu-games, or \emph{educational games}, are a type of serious game, as defined by Ritterfeld et al.: "any form of interactive computer-based game software for one or multiple players to be used on any platform and that has been developed with the intention to be more than entertainment"\citep{Ritterfeld}. The main purpose of an edu-game is to educate its users on some topic or to assist in learning situations by providing a platform for repetition of information gained elsewhere, e.g. through a museum exhibition.

%----------------------------------------------------------------------------------------
%	SECTION 4
%----------------------------------------------------------------------------------------

\section{Gamification and Game Elements}

"Gamification is the use of design elements characteristic for games in non-game contexts" \citep{Deterding}. While an edu-game does not qualify as a non-game context, the number of game elements used can differ from game to game. An edu-game with few game elements may blur the line between a serious game and a gamified non-game application.

There are many possible game elements that can be included to make a game more entertaining and engaging to its players. Reeves and Read have identified "Ten ingredients of Great Games"\citep{Reeves}: Self-representation with avatars; three-dimensional environments; narrative context; feedback; reputations, ranks, and levels; marketplaces and economies; competition under rules that are explicit and enforced; teams; parallel communication systems that can be easily configured; time pressure.  A game that implements one or more of these elements is likely to provide a positive experience to its players.