% Chapter 1

\chapter{Introduction}

\label{Chapter1}

\lhead{Chapter 1. \emph{Introduction}}

%----------------------------------------------------------------------------------------
%	SECTION 1
%----------------------------------------------------------------------------------------

\section{Introduction}

The goal of this paper is to plan the improvement of the existing Flora Danica application, a modifiable edu-game built on the Kivy framework\citep{Kivy}, designed for use on large multi-touch displays in informal learning situations. The paper will look into previous research and work on related topics such as edu-games, gamification, game elements, multi-touch wall displays and tabletops, and will use concepts from these areas to improve the existing application.

The first chapter provides an introduction to the paper, the existing application and edu-games in general. The second chapter explores and discusses previous work on related topics. The third chapter proposes a solution in possible extensions to the application. It also describes the planned process for conducting the user studies. The final chapter contains final discussion as well as a summary of the paper.

%----------------------------------------------------------------------------------------
%	SECTION 2
%----------------------------------------------------------------------------------------

\section{Existing application}

The existing Flora Danica application that is being extended is a Python application built partially on the Kivy framework\citep{Kivy}. We are going to focus on improving the part built on the Kivy framework, which is a simple quiz game that can be played by two players either as a team ("team mode") or against each other ("versus mode").

During the game, the players will be presented with a series of questions, each with two alternatives for the right answer. In versus mode, each player answers the question separately, touching their answer button with one finger for the first alternative and two fingers for the second alternative. In team mode, the player to the left touches their answer button for the first alternative, while the player on the right touches their button for the second alternative.

%----------------------------------------------------------------------------------------
%	SECTION 3
%----------------------------------------------------------------------------------------

\section{Learning Applications}

Learning applications are applications whose goal is to assist users in the learning process, either through teaching a new topic or helping to remember or cement previously gained knowledge. Using learning applications provides some benefits over traditional learning methods such as lectures or paper hand outs, which are covered in the next chapter. Learning applications are often created as games or with aspects from gamification, which can add to the appeal and effect of using these methods over traditional methods.

%----------------------------------------------------------------------------------------
%	SECTION 4
%----------------------------------------------------------------------------------------

\section{Gamification}

"Gamification is the use of design elements characteristic for games in non-game contexts" \citep{Deterding}. By including aspects from the field of gamification into non-game contexts – e.g. learning applications – users can feel more motivated to complete the tasks at hand.

A common example of gamification is the addition of "achievements" where you are awarded for completing tasks or for your performance in the form of "badges" in the application. Other forms include dividing the content into something akin to "levels" found in many games, requiring the user to unlock new "levels" by completing previous ones.