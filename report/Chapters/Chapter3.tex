% Chapter 3

\chapter{Proposed Solution}

\label{Chapter3}

\lhead{Chapter 3. \emph{Proposed Solution}}

%----------------------------------------------------------------------------------------
%	SECTION 1
%----------------------------------------------------------------------------------------

\section{Introduction}

This chapter describes the proposed solution. It starts by discussing architectural decisions and collection and generation of data. It also considers options to increase motivation and challenge in the game through new game modes and variable difficulty. Lastly it discusses the addition of achievements, badges and high scores as additional motivation for the players.

%----------------------------------------------------------------------------------------
%	SECTION 2
%----------------------------------------------------------------------------------------

\section{Architecture}

The Flora Danica application is used as a base for several other applications, such as those of the Norwegian Deaf Museum and of the Norwegian States Archive in Trondheim. Therefore it is important that implementation is kept general and easily modifiable. Features beyond the absolute core should be optional, as all features may not be interesting or suited for all variations of the application.

To make it easy to opt into or out of certain features, it makes sense to modularize them. By moving each feature into its own module, it's easy for an application to add or remove features as they gain or lose relevancy. Keeping features in separate modules has the added benefit of making them easy to maintain and modify.

%----------------------------------------------------------------------------------------
%	SECTION 3
%----------------------------------------------------------------------------------------

\section{Data Tracking and Tagging}

In the existing application, data is gathered about where users touch, the time spent to answer questions and the number of correctly answered questions. By tagging questions and tasks with various categories, we are able to use these tags along with the collected data to extend the application with further game elements. The tags can be used to differentiate questions by difficulty or category (e.g. "World War II", "Cold War", "21st Century").

%----------------------------------------------------------------------------------------
%	SECTION 4
%----------------------------------------------------------------------------------------

\section{Game Modes, Randomization and Variable Difficulty}

In the current version of the application, each playthrough will present the players with all available questions. For an instance of the application that is placed at the end of a museum, this approach lets the application summarize all the information the museum visitors have or should have gained during their visit. However, this approach may feel more like an exam than a game, which may make it less appealing to play.

By supporting additional game modes, we can allow for more varied and entertaining play. In addition to the current mode that presents all available questions, questions can be divided into groups of 5-10 questions based on tags to allow level or round based play. Each round presents a given number of questions randomly selected from the appropriate set that the players have to answer in a set amount of time. Each round increases the difficulty by selecting more difficult questions and/or decreasing the available time for the round. This eases the players into the game, but increases the challenge gradually, which makes for a more entertaining experience overall. By selecting the questions semi-randomly, and randomly selecting the order of the alternatives, we also place the difficulty on knowing the correct answers rather than knowing whether the left or right alternative is the correct for each question (with all questions appearing each play, in the same order).

Another way of increasing the difficulty and challenge is to introduce additional alternatives to each question. A question with three alternatives can be considered more difficult than one with two, provided that all the alternatives sound plausible. This also makes guessing the correct answer more difficult. It is however important that players playing together in coop mode know which alternatives each of them should answer. Therefore it makes sense to differentiate difficulties in increments of two alternatives, and because six alternatives is a bit excessive for relatively simple questions, coop mode should be limited to two difficulties: easy (two alternatives) and hard (four alternatives). The left player can then answer the leftmost alternatives with one or two fingers respectively, and vice versa for the right player. Versus mode can have any number of alternatives as the chosen alternative depends on the number of fingers (touch points) used to answer, as long as the input field is large enough to accommodate enough fingers. However, four fingers is the maximum amount of fingers a player can easily use simultaneously in a small area, so it makes sense to use this as an upper limit. It also makes sense to use the same difficulty levels as coop mode, with two and four alternatives. Additionally, versus mode can include a medium difficulty, with three alternatives.

This leads to three main changes: Division of questions into rounds or levels, semi-randomized choice of questions for each round, and variable difficulty with harder difficulties having more alternatives per question. The player(s) will then make three choices before playing:
\begin{enumerate}
	\item Play together (coop mode) or against each other (versus mode)?
	\item Play through all questions (summary mode) or timed rounds (level mode)?
	\item What difficulty would you like to play?
\end{enumerate}

%----------------------------------------------------------------------------------------
%	SECTION 5
%----------------------------------------------------------------------------------------

\section{Goals: High Scores and Achievements}

The Flora Danica application is designed for use in informal learning situations, and will typically only be used once by each player at the end of a museum visit or similar situation. The game will test the players on the knowledge they gained during their visit, but the only goal in the existing application is getting as many points as possible by answering as many questions correctly as they can. Beyond this, however, there is currently nothing more.

By implementing achievements we give the players additional goals to strive towards. However, because the games are not connected to any external achievement system, it is important that we give the achievements some additional meaning. By creating a "Hall of Fame", where achievements are listed with all players who have earned them, players can earn recognition from future players. Additionally, the institutions hosting the game can give out-game rewards for getting certain achievements. Whether to offer rewards and what those rewards are would be up to the individual institution, but examples could be something like a diploma, a trinket from their gift shop or a free ticket for two people to come back to the museum at a later time. Examples of achievements are:
\begin{itemize}
	\item \emph{Flawless}: Finish summary mode with no incorrect answers
	\item \emph{Sprinter}: Finish each level in level mode in less than 30 seconds
\end{itemize}

In addition to the achievements, we can use the tracked data to keep scores. Each player or team of players receive points based on correct answers and how fast they were able to answer. A "High Scores" list compiles the best of these scores for each game mode, to historically rank the best players.

While the two aforementioned methods give players long-term recognition for good performance, players might also want something to distinguish their performances from each other in one particular game (e.g. if they play several games). Lets call these short-term, per-game achievements for badges. Badges represent smaller achievements and promote competition between players playing together at the same time. By giving badges to all participating players, we also help players feel included. These small awards should be humoristic, and don't necessarily need to award a positive aspect such as being best, but can "award" players for most incorrect answers or most hits outside their answer box.

An example of a popular game that uses this approach is the multiplayer online battle arena game \emph{Heroes of Newerth}\citep{Newerth}. In their \emph{Mid Wars} game mode, a mode much more focused on fun than the competitive aspect, each player will receive two such badges at the end of each game. The badges can be based on a large number of different statistics, such as hero damage or kills, buildings destroyed, gold earned and so on\citep{MidWars}. There are also some number of badges that reward feats that may not be particularly impressive, such as having no kills. Examples of such badges for our games could be:
\begin{itemize}
	\item \emph{Veteran}: No wrong answers on questions in the "World War II" category
	\item \emph{Pacifist}: No correct answers on questions in the "World War II" category
	\item \emph{Poor Aim}: Most presses outside of answer box
\end{itemize}

The first two badges utilize the category tags, while the \emph{Poor Aim} badge would use the touch data to determine how many presses were made outside of the designated answer box on each side of the screen. It would assume that touches on the left side of the screen was made by the left player an vice versa.