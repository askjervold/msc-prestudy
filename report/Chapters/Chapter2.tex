% Chapter 2

\chapter{Related Work}

\label{Chapter2}

\lhead{Chapter 2. \emph{Related Work}}

%----------------------------------------------------------------------------------------
%	SECTION 1
%----------------------------------------------------------------------------------------

\section{Introduction}

This chapter describes and compares previous work done in related fields. We will look into studies on informal learning, collaboration, game design and motivation, and relate their findings to each other to gain an overview of what may or may not work for our project.

%----------------------------------------------------------------------------------------
%	SECTION 2
%----------------------------------------------------------------------------------------

\section{Edu-games and Informal Learning}

Ardito et al.\citep{Ardito} performed a study to validate an educational format split into three phases: \emph{symbolic}, \emph{active} and \emph{iconic}. They created an educational game named History-Puzzle for use in the iconic learning phase as a supplement to more traditional learning methods in the other phases, and found it to be an effective approach. In the iconic learning phase, learners rely on iconic knowledge representation such as images of the concepts to be learned.

In their study, students first received a lecture on Roman history, followed by a field trip to a historic area where they more closely experienced what they had learned during the lecture. The following week they used History-Puzzle, where they had to use the knowledge they had gained to build factual sentences about the subject from several alternatives presented.

A multiple-choice test given to the students before and after use of the application showed that students had more correct answers after using the application, supporting the theory that such learning applications are able to take the role of an iconic learning medium. They make no claims as to the effect of this approach given a different sequence of learning phases (e.g. iconic first, followed by symbolic then active), but their findings show that with the given structure, their approach to the symbolic phase is indeed effective.

Antle et al.\citep{AntleFutura} chose to focus their game not on teaching users about specific concepts, but rather to give users a better understanding of familiar concepts and facilitate behavioral change through a shift in awareness. Their real-time simulation game Futura let players experience the difficulty and trade-offs involved in trying to balance the needs of a growing population with those of the environment, rather than simply presenting them as factual statements.

In their study, they found that a significant amount of their users felt that they learned from the experience, while contemporary studies of the time showed that simply informing people of issues of public concern was not enough to change their behaviors. This speaks well for the effect of edu-games in providing understanding of difficult topics that other learning formats have trouble in teaching.

Horn et al.\citep{BAT} created an edu-game that would function as an interactive exhibition in natural history museums. They wanted to make sure their application would be perceived as "a real game, as opposed to an activity with superficial game-like elements". To achieve this, they focused on making their game fun and easy to play, while still requiring players to learn something, as correct answers were required to progress in the game.

While their game succeeded in engaging visitors of the museum in the exhibit, they did not succeed in their goal of creating a game that allowed for open-ended exploration where players could explore and answer their own questions, and found that visitors' motivation for playing the game was related more to the challenge in the game than to interest in learning about the topic.

%----------------------------------------------------------------------------------------
%	SECTION 3
%----------------------------------------------------------------------------------------

\section{Multi-Touch Platform and Collaboration}

Ardito et al. found that engaging the students in this type of learning activity increased inclusion among the students, and even the most timid of the students participated actively in the collaboration. While in some groups the students divided the tasks between them and worked more or less individually, in most of the groups all students were involved in solving the puzzle. The researchers attribute the high degree of collaboration to the multi-touch display format, and found that it encourages group activities.

Antle et al. chose to split the sources of user interaction spatially on their multi-touch platform, physically preventing one player from controlling the entire game. Although there were cases of players leaning over to another player's toolbar to interact, most players collaborated by controlling their own toolbars and discussing mutual decisions. Some of their players reported that they learned from playing together, because they were able to help each other. Playing with others also allowed individual players to focus on learning about one area (e.g. food production) per playthrough, with the option of playing a different role on subsequent playthroughs.

Snibbe and Raffle\citep{Snibbe} defined the concept of \emph{social scalability} as one of several guidelines for multi-user interaction in interactive exhibits. They state that "if the exhibit fits more than one person, it must work with more than one person", and that "users’ engagement and satisfaction should become richer as more people interact". Horn et al. developed their game for what they deemed a relatively small surface, which somewhat limited the social scalability in that the amount of physical collaboration (simultaneous manipulation of the game itself) possible was limited. They did however find that users collaborated verbally a lot in addition to the physical collaboration. In fact, well over 90\% of verbal communcation between the observed users was related to the interactive exhibition (the game), much of which was discussion of pacing, possible answers or feedback on the progress.

Jakobsen and Hornbæk\citep{Jakobsen} conducted an exploratory study to achieve a better understanding of how groups collaborate around large, wall-sized multi-touch displays. They found that the large multi-touch display supported both parallel and joint work due to allowing input from multiple users simultaneously and due to the large display area. This benefit was apparent in the studies performed by both Ardito et al. and Horn et al., where players could either work together on the same piece of the puzzle or on separate pieces simultaneously.

Birnholtz et al.\citep{Birnholtz} conducted a similar study to that of Jakobsen and Hornbæk in 2007, exploring input configurations for large displays, comparing the use of a single, shared mouse and one mouse per user. They also found that multiple input sources allowed for a greater degree of parallel work, while a single mouse allowed a single user to dominate the task. While multiple inputs increased the amount of parallel work, it also decreased the quality of the discussion. Jakobsen and Hornbæk, however, found that this issue was lessened when the input source was touch rather than a mouse, experiencing "higher levels of awareness and a higher incidence of verbal shadowing".

%----------------------------------------------------------------------------------------
%	SECTION 4
%----------------------------------------------------------------------------------------

\section{Gamification, Game Elements and Motivation}

Reeves and Read\citep{Reeves} identified ten ingredients of great games, as mentioned in the previous chapter. The list is as follows: Self-representation with avatars; three-dimensional environments; narrative context; feedback; reputations, ranks, and levels; marketplaces and economies; competition under rules that are explicit and enforced; teams; parallel communication systems that can be easily configured; time pressure. They state that a good game usually uses some variation of the ten ingredients they present, though not necessarily all of them.

While their book focuses on long-term use of games for work, many of the same principles apply to edu-games for our purposes, such as narrative context, feedback, reputations/ranks/levels, competition and teams, as well as time pressure. Self-representation with avatars and marketplaces and economies are features better suited for games played over longer periods of time than what edu-games that are part of a museum exhibit typically are, and communication systems are more important in games where the players are not co-located. Three-dimensional environments provide several benefits (such as ease-of-use and spatial awareness), and could be used successfully in smaller contexts, but are typically used for larger games.

Malone\citep{Malone} conducted a study on what makes computer games captivating and fun, and how to use these elements in learning situations, particularly with edu-games. He identified three main categories that affect motivation to play games and learn through them: challenge, fantasy and curiosity. Challenge relies on goals and uncertain outcome and is related to self-esteem.

Antle et al. deliberately made their game challenging to the point where it was unlikely that players would be able to win on their first playthrough. Similarly, Ardito et al. included superfluous/false sentence fragments for their sentence matching puzzle to increase the difficulty level to keep the game challenging and intriguing. Malone points out that keeping the games challenging is important to keep players interested, as achieving success in challenging games and activities raises self-esteem and makes people feel better. It is however important to not make the games too challenging, as repeated failure in challenging games can lower players' self-esteem and decrease their desire to play the game.

Goals not only introduce challenge (players want to reach the goals), but making the goals clear and providing feedback on the players' progress towards the goals makes it easier to stay motivated when the challenge proves too big. Performance feedback can keep the challenge inviting rather than discouraging, e.g. "Not quite, but you're getting close!". Malone describes multiple levels of goals, where the levels of goals can either differ in how difficult it is to reach the goal (e.g. clear the first puzzle vs. clear all the puzzles), or where the higher level goals involve completing the lower level goals "better" (e.g. clear the first puzzle with fewer mistakes or in shorter time). Scorekeeping is a common way of introducing higher level goals: the low level goal is to complete a puzzle, while the higher level goals is to complete the same puzzle with a higher score.

Variable difficulty is another way of reducing the risk of a too high degree of challenge. By allowing different levels of difficulty, either chosen by the player or automatically chosen by the game based on performance or perceived skill, players can face a challenge tailored to their capabilities that will keep the game engaging but not discouraging. This variable difficulty can also be achieved through levels. By dividing game content into increasingly difficult levels, as seen in Horn et al.'s Build-a-Tree, players are able to gradually increase the challenge and stop when the game becomes too challenging.

One way of implementing uncertain outcome is through randomness. In Ardito et al.'s History-Puzzle, for each round, the questions and answers were selected randomly from a list of over twice as many possibilities as the players were presented. Additionally, the tiles containing the questions and answers (or each half of the sentence) are randomly placed in their respective areas. This randomness gives replayability in addition to the aforementioned challenge, because you can get different questions each time you play, and one cannot simply memorize the position of the correct answers to get the maximum score, but actually need to learn the answers. 

Uncertain outcome can also be achieved with hidden information, such as in a guessing game. Malone found that hidden information can provoke curiosity as well as contribute to the challenge of the game. The concept of curiosity can be divded into \emph{sensory curiosity} and \emph{cognitive curiosity}. As Malone puts it, "sensory curiosity involves the attention-attracting value of changes in the light, sound, or other sensory stimuli of an environment". An example of this is the changing color of the environment in Antle et al.'s Futura, which should cause players to reflect on the status of the game. Cognitive curiosity is related to bringing \emph{completeness} and \emph{consistency} to what we know. By presenting the player with just the right amount of information, they may be motivated to seek out more information to answer their questions and "complete" their knowledge or understanding. An example of this is seen in Horn et al.'s Build-a-Tree, where correct sub-trees are labeled with synapomorphies as they are constructed, urging players to complete more sub-trees to learn what traits other species share.