% Chapter 2

\chapter{Related Work}

\label{Chapter2}

\lhead{Chapter 2. \emph{Related Work}}

%----------------------------------------------------------------------------------------
%	SECTION 1
%----------------------------------------------------------------------------------------

\section{Introduction}

This chapter describes previous work done in related fields. The relevant topics such as edu-games, multi-touch platforms and collaboration are often intertwined, and as such the studies typically cover several or all of these topics.

%----------------------------------------------------------------------------------------
%	SECTION 2
%----------------------------------------------------------------------------------------

\section{Edu-games and Informal Learning}

Ardito et al.\citep{Ardito} performed a study to validate an educational format split into three phases: \emph{symbolic}, \emph{active} and \emph{iconic}. They created an educational game named History-Puzzle for use in the iconic learning phase as a supplement to more traditional learning methods in the other phases, and found it to be an effective approach. In the iconic learning phase, learners rely on iconic knowledge representation such as images of the concepts to be learned.

In their study, students first received a lecture on Roman history, followed by a field trip to a historic area where they more closely experienced what they had learned during the lecture. The following week they used History-Puzzle, where they had to use the knowledge they had gained to build factual sentences about the subject from several alternatives presented.

A multiple-choice test given to the students before and after use of the application showed that students had more correct answers after using the application, supporting the theory that such learning applications are able to take the role of an iconic learning medium. They make no claims as to the effect of this approach given a different sequence of learning phases (e.g. iconic first, followed by symbolic then active), but their findings show that with the given structure, their approach to the symbolic phase is indeed effective.

Antle et al.\citep{AntleFutura} chose to focus their game not on teaching users about specific concepts, but rather to give users a better understanding of familiar concepts and facilitate behavioral change through a shift in awareness. Their real-time simulation game Futura let players experience the difficulty and trade-offs involved in trying to balance the needs of a growing population with those of the environment, rather than simply presenting them as factual statements.

In their study, they found that a significant amount of their users felt that they learned from the experience, while contemporary studies of the time showed that simply informing people of issues of public concern was not enough to change their behaviors. This speaks well for the effect of edu-games in providing understanding of difficult topics that other learning formats have trouble in teaching.

Horn et al.\citep{BAT} created an edu-game that would function as an interactive exhibition in natural history museums. They wanted to make sure their application would be perceived as "a real game, as opposed to an activity with superficial game-like elements". To achieve this, they focused on making their game fun and easy to play, while still requiring players to learn something, as correct answers were required to progress in the game.

While their game succeeded in engaging visitors of the museum in the exhibit, they did not succeed in their goal of creating a game that allowed for open-ended exploration where players could explore and answer their own questions, and found that visitors' motivation for playing the game was related more to the challenge in the game than to interest in learning about the topic.

%----------------------------------------------------------------------------------------
%	SECTION 3
%----------------------------------------------------------------------------------------

\section{Multi-Touch Platform and Collaboration}

Ardito et al. found that engaging the students in this type of learning activity increased inclusion among the students, and even the most timid of the students participated actively in the collaboration. While in some groups the students divided the tasks between them and worked more or less individually, in most of the groups all students were involved in solving the puzzle. The researchers attribute the high degree of collaboration to the multi-touch display format, and found that it encourages group activities.

Antle et al. chose to split the sources of user interaction spatially on their multi-touch platform, physically preventing one player from controlling the entire game. Although there were cases of players leaning over to another player's toolbar to interact, most players collaborated by controlling their own toolbars and discussing mutual decisions. Some of their players reported that they learned from playing together, because they were able to help each other. Playing with others also allowed individual players to focus on learning about one area (e.g. food production) per playthrough, with the option of playing a different role on subsequent playthroughs.

Snibbe and Raffle\citep{Snibbe} defined the concept of \emph{social scalability} as one of several guidelines for multi-user interaction in interactive exhibits. They state that "if the exhibit fits more than one person, it must work with more than one person", and that "users’ engagement and satisfaction should become richer as more people interact". Horn et al. developed their game for what they deemed a relatively small surface, which somewhat limited the social scalability in that the amount of physical collaboration (simultaneous manipulation of the game itself) possible was limited. They did however find that users collaborated verbally a lot in addition to the physical collaboration. In fact, well over 90\% of verbal communcation between the observed users was related to the interactive exhibition (the game), much of which was discussion of pacing, possible answers or feedback on the progress.

Jakobsen and Hornbæk\citep{Jakobsen} conducted an exploratory study to achieve a better understanding of how groups collaborate around large, wall-sized multi-touch displays. They found that the large multi-touch display supported both parallel and joint work due to allowing input from multiple users simultaneously and due to the large display area. This benefit was apparent in the studies performed by both Ardito et al. and Horn et al., where players could either work together on the same piece of the puzzle or on separate pieces simultaneously.

Birnholtz et al.\citep{Birnholtz} conducted a similar study to that of Jakobsen and Hornbæk in 2007, exploring input configurations for large displays, comparing the use of a single, shared mouse and one mouse per user. They also found that multiple input sources allowed for a greater degree of parallel work, while a single mouse allowed a single user to dominate the task. While multiple inputs increased the amount of parallel work, it also decreased the quality of the discussion. Jakobsen and Hornbæk, however, found that this issue was lessened when the input source was touch rather than a mouse, experiencing "higher levels of awareness and a higher incidence of verbal shadowing".

%----------------------------------------------------------------------------------------
%	SECTION 4
%----------------------------------------------------------------------------------------

\section{Gamification and Game Elements}

Reeves and Read\citep{Reeves} identified ten ingredients of great games, as mentioned in the previous chapter. The list is as follows: Self-representation with avatars; three-dimensional environments; narrative context; feedback; reputations, ranks, and levels; marketplaces and economies; competition under rules that are explicit and enforced; teams; parallel communication systems that can be easily configured; time pressure. They state that a good game usually uses some variation of the ten ingredients they present, though not necessarily all of them.

While their book focuses on long-term use of games for work, a lot of the same principles apply to edu-games for our purposes. Self-representation with avatars is related to familiarity, social interaction and user engagement. The engagement aspect comes from customizing an avatar, which users feel connected to, and is more suitable for games played over a longer period of time than those at museum exhibits. The familiarity aspect (avatars represent people, which we know how to relate to), however, can be applied to shorter sessions of play as well. Three-dimensional environments are appealing because we usually know how they work (from real-life experience), meaning no (or at least less) instruction is necessary. They also facilitate spatial awareness, which is especially important for repeated use of elements located in specific places in the virtual space.

Narrative contexts appeal to human's fondness for stories, while guiding players through the game. They create excitement and engagement, which increases the both chance that people will want to play the game, and that they will keep playing until finished. Stories also aid memory, making it more likely that players will retain the information they learn through the game.



%Both of the previously mentioned research teams chose to develop their learning application as a game, as educational games have proven to have many benefits\citep{Ardito}. Both teams found their subjects to enjoy their game, which increased their desire to play and thereby learn.
%
%Antle et al. deliberately made their game challenging to the point where it was unlikely that players would be able to win on their first playthrough. Similarly, Ardito et al. included superfluous/false sentence fragments for their sentence matching puzzle to increase the difficulty level to keep the game challenging and intriguing. Malone\citep{Malone} points out that keeping the games challenging is important to keep players interested, as achieving success in challenging games and activities makes people feel better, and failure in challenging games can decrease players' desire to play the game.
%
%Malone also mentions randomness as an aspect that heightens player interest, due to the uncertainty of outcomes. In Ardito et al.'s History-Puzzle, for each round, the questions and answers were selected randomly from a list of over twice as many possibilities as the players were presented. Additionally, the tiles containing the questions and answers (or each half of the sentence) are randomly placed in their respective areas. This randomness gives replayability in addition to the aforementioned challenge, because you can get different questions each time you play, and one cannot simply memorize the position of the correct answers to get the maximum score, but actually need to learn the answers.