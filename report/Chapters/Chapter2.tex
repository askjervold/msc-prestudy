% Chapter 2

\chapter{Related Work}

\label{Chapter2}

\lhead{Chapter 2. \emph{Related Work}}

%----------------------------------------------------------------------------------------
%	SECTION 1
%----------------------------------------------------------------------------------------

\section{Introduction}

This chapter describes previous work that has been done in related fields.

%----------------------------------------------------------------------------------------
%	SECTION 2
%----------------------------------------------------------------------------------------

\section{Existing application}

The existing Flora Danica application that is being extended is a Python application built partially on the Kivy framework. We are going to focus on improving the part built on the Kivy framework, which is a simple quiz game that can be played by two players either as a team ("team mode") or against each other ("versus mode").

During the game, the players will be presented with a series of questions, each with two alternatives for the right answer. In versus mode, each player answers the question separately, touching their answer button with one finger for the first alternative and two fingers for the second alternative. In team mode, the player to the left touches their answer button for the first alternative, while the player on the right touches their button for the second alternative.

%----------------------------------------------------------------------------------------
%	SECTION 3
%----------------------------------------------------------------------------------------

\section{Learning Applications}

Many studies have been done on the merits of learning applications, and the results are generally positive. Ardito et al.\citep{Ardito} found that learning applications are a good supplement to traditional learning, and can help in the final phase of learning, where learners rely on iconic knowledge representation such as images to solidify things learned previously. In their study, students received a lecture on Roman history, then went on a field trip to an old area where they more closely experienced what they had learned during the lecture. The following week they used a learning application where they had to use the knowledge they had gained to create factual sentences about the subject. A multiple-choice test given to the students before and after use of the application showed that students had more correct answers after using the application.

%----------------------------------------------------------------------------------------
%	SECTION 4
%----------------------------------------------------------------------------------------

\section{Multi-User Applications}

This section describes findings regarding research on multi-user applications, i.e. what is important to focus on for applications specifically developed as multi-user apps.

%----------------------------------------------------------------------------------------
%	SECTION 5
%----------------------------------------------------------------------------------------

\section{Gamification}

This section describes findings regarding research on gamification, such as benefits and important concepts. E.g. different kinds of motivation.

\subsection{Achievements/Badges}

Variations of achievement implementations.

%----------------------------------------------------------------------------------------
%	SECTION 6
%----------------------------------------------------------------------------------------

\section{User Studies and Analytics}

This section describes findings regarding research on user study techniques, usage analytics and user experience.