% Chapter 2

\chapter{Related Work}

\label{Chapter2}

\lhead{Chapter 2. \emph{Related Work}}

%----------------------------------------------------------------------------------------
%	SECTION 1
%----------------------------------------------------------------------------------------

\section{Introduction}

This chapter describes previous work that has been done in related fields. The relevant topics such as learning applications, collaboration, multi-touch platforms and gamification are often intertwined, and as such the studies typically cover several or all of these topics.

%----------------------------------------------------------------------------------------
%	SECTION 2
%----------------------------------------------------------------------------------------

\section{Related Work}

Ardito et al.\citep{Ardito} wished to validate an educational format split into three phases: \emph{symbolic}, \emph{active} and \emph{iconic}. They found using learning applications in the iconic learning phase as a supplement to more traditional learning methods in the other phases to be an effective approach. In the iconic learning phase, learners rely on iconic knowledge representation such as images of the concepts to be learned.

In their study, students first received a lecture on Roman history, followed by a field trip to a historic area where they more closely experienced what they had learned during the lecture. The following week they used a learning application where they had to use the knowledge they had gained to build factual sentences about the subject from alternatives in an educational game.

A multiple-choice test given to the students before and after use of the application showed that students had more correct answers after using the application, supporting the theory that such learning applications are able to take the role of an iconic learning medium. They make no claims as to the effect of this approach given a different sequence of learning phases (e.g. iconic first, followed by symbolic then active), but their findings show that with the given structure, their approach to the symbolic phase is indeed effective.

Antle et al.\citep{AntleFutura} chose to focus their application not on teaching users about specific concepts, but rather to give users a better understanding of familiar concepts and facilitate behavioral change through a shift in awareness. Their real-time simulation game Futura let players experience the difficulty and trade-offs involved in trying to balance the needs of a growing population with those of the environment, rather than simply presenting them as factual statements.

In their study, they found that a significant amount of their users felt that they learned from the experience, while contemporary studies of the time showed that simply informing people of issues of public concern was not enough to change their behaviors. This speaks well for the effect of this type of learning applications.

Ardito et al. found that engaging the students in this type of learning activity increased inclusion among the students, and even the most timid of the students participated actively in the collaboration. While in some groups the students divided the tasks between them and worked more or less individually, in most of the groups all students were involved in solving the puzzle. The researchers attribute the high degree of collaboration to the multi-touch display format, and found that it encourages group activities.

Antle et al. chose to split the sources of user interaction spatially on their multi-touch platform, physically preventing one player from controlling the entire game. Although there were cases of players leaning over to another player's toolbar to interact, most players collaborated by controlling their own toolbars and discussing mutual decisions. Some of their players reported that they learned from playing together, because they were able to help each other. Playing with others also allowed individual players to focus on learning about one area (e.g. food production) per playthrough, with the option of playing a different role on subsequent playthroughs.


%\section{Multi-User Applications}
%This section describes findings regarding research on multi-user applications, i.e. what is important to focus on for applications specifically developed as multi-user apps.

%\section{Gamification}
%This section describes findings regarding research on gamification, such as benefits and important concepts. E.g. different kinds of motivation.