\chapter{Conclusion}

\label{Chapter5}

\lhead{Chapter 5. \emph{Conclusion}}

%----------------------------------------------------------------------------------------
%	SECTION 1
%----------------------------------------------------------------------------------------

\section{Conclusion}

Throughout the paper we have looked into previous work on edu-games and interactive museum exhibits, game elements and motivation. We have seen that edu-games can be effective tools for learning in informal learning situations and for facilitating attitude change where traditional information channels struggle. We have also seen that finding the right balance between a game and an informational tool can be difficult, and that it is important to realize that the motivations for playing the game in some cases may be the game itself rather than the learning it supports.

We have also seen that large multi-touch displays not only encourage group activities and collaboration, but that large multi-touch exhibits are expected to support interaction from multiple simultaneous users. Furthermore, users find these kind of exhibits fun and easy to interact with, and having several input sources leads to less frustration.

Finally, we have seen that successful games share a lot of common elements, and that motivation stems from three main elements: challenge, fantasy and curiosity. Challenge derives from goals and uncertain outcome, which can be achieved through various means, such as randomness and hidden information. Challenge and goals affect self-esteem, and it is important that they are balanced, e.g. through variable difficulty. While achievements is a very popular game element, it is important that they are implemented in a way that they actually increase motivation rather than function as a detriment, reducing the game to a simple grind for achievements.

Moving forward, the existing application will be extended with additional game modes, variable difficulty levels and randomization to increase the challenge and the fun. Additionally, the players will be motivated to perform their best with new goals: getting in the hall of fame and topping the high scores lists. The extended application will be tested in parallel with the current version by separate groups to compare them on aspects of fun, motivation and learning. Through questionnaires, observations and analysis of background data, we should be able to assess what elements are effective in reaching our goals of a fun, motivating and educational game.